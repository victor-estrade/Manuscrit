%!TEX root = ../thesis.tex
%*******************************************************************************
%****************************** Third Chapter **********************************
%*******************************************************************************
\chapter{Behind the scene}
\label{chap:behind}
% **************************** Define Graphics Path **************************
\ifpdf
    \graphicspath{{Chapter7/Figs/Raster/}{Chapter7/Figs/PDF/}{Chapter7/Figs/}}
\else
    \graphicspath{{Chapter7/Figs/Vector/}{Chapter7/Figs/}}
\fi

% The detailed plan is prospective at this stage


% Find a few meta parameter to evaluate the performances, and explain why it varies like that
% \begin{itemize}
% \item "learning curve" depending on number of training domain seen
% \item number of guided dropout block added
% \item show some experiments on simplified grid without production disconnections
% \item Une expérience d'apprentissage que j'aimerai voir et qui me semblerait être une bonne illustration de ce que permet l'architecture GD: montrer qu'il est possible d'apprendre incrémentalement ou séquentiellement de nouvelles conditions unitaires par rapport à un modèle déjà appris autour de conditions existantes, avec d'aussi bonnes perfs qu'un apprentissage où tout est réappris jointement sur un gros volume de données (Antoine)
% \end{itemize}

\section{Description of the data}