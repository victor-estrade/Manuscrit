%!TEX root = ../thesis.tex
%*******************************************************************************
%****************************** sixth Chapter **********************************
%*******************************************************************************
\chapter{Discussion and conclusion}
\label{chap:conclusion}
% **************************** Define Graphics Path **************************
\ifpdf
    \graphicspath{{Chapter6/Figs/Raster/}{Chapter6/Figs/PDF/}{Chapter6/Figs/}}
\else
    \graphicspath{{Chapter6/Figs/Vector/}{Chapter6/Figs/}}
\fi


\victor{L'idée que c'est un graphical model mais les z sont dans le simulateur}
\victor{L'idée que si c'est sans params de nuisance alors c'est trop facile. On aurait que l'erreur statistique. On ne peut l'améliorer qu'en améliorant le classifieur. D'où vient la variance alors ?}
\victor{Sur les vrais données on doit utiliser une méthode numérique}
\victor{Quel est le budget accessible ? Combien coute la méthode normale en théorie et dans la vrai vie ? On refait vraiment tourner la simulation à chaque fois ? Combien coute ma méthode ?}
\victor{Neural network as Estimator \& estimate its variance}


\section{Calibration} % (fold)
\label{sec:calibration}

\subsection{Common calibration} % (fold)
\label{sub:common_calibration}


Multiple study using data comming from the same experiment should have the same nuisance parameter estimation.



\subsection{Update calibration} % (fold)
\label{sub:update_calibration}

Should we use the data to improve inference on the nuisance parameters ?

Technically maximum likelihood inference is doing it through maximizing the likelihood.
\victor{Oui ça semble être une totologie mais en fait c'est pas évident}


\section{Further works} % (fold)
\label{sec:further_works}

Ideas for the next steps

\subsection{Real simulator} % (fold)
\label{sub:real_simulator}

Use a real simulator instead of a trick to get a fast simulator.
Improve it with minong gold idea if possible.
Add calibration steps in the simulation.

This way the entire workflow is under control and monitoring.



\subsection{Estimate the progression margin} % (fold)
\label{sub:estimate_the_progression_margin}

Ideas to find a method to measure the maximum information that can be extracted from the given data.






