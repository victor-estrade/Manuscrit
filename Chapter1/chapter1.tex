%!TEX root = ../thesis.tex
%*******************************************************************************
%*********************************** First Chapter *****************************
%*******************************************************************************

\chapter{Data-driven science}  %Title of the First Chapter
\label{chap:intro_phy}
\ifpdf
    \graphicspath{{Chapter1/Figs/Raster/}{Chapter1/Figs/PDF/}{Chapter1/Figs/}}
\else
    \graphicspath{{Chapter1/Figs/Vector/}{Chapter1/Figs/}}
\fi

Contexte et position du problème du point de vue de la physique.

\section{Background}


\subsection{High energy physics} % (fold)
\label{sub:high_energy_physics}

\subsection{CERN} % (fold)
\label{sub:cern}


\subsection{Systematics} % (fold)
\label{sub:systematics}

\section{Motivation} % (fold)
\label{sec:motivation}

Limitations des méthodes actuelles

\subsection{Accuracy} % (fold)
\label{sub:accuracy}


\subsection{computation} % (fold)
\label{sub:computation}

\subsection{Understanding and methodology} % (fold)
\label{sub:understanding_and_methodology}

Levée de confusion entre mesure et découverte.

Évaluation quantitative empirique des résultats => test statistique "non parametrique"

+ justifications théorique


\section{OLD} % (fold)



\section{Before Machine learning}

How it was done before machine learning

\section{With machine learning}

How it is done with machine learning


\section{Estimating a cross section} % (fold)
\label{sec:estimating_a_cross_section}

We are studying a stochatic phenomenon which generative process is described as :

\begin{equation}
	\label{eq:mixture_model}
	p(x|m) = m p(x|s) + (1-m) p(x|b)
\end{equation}
where $x$ is the set of observable features of the studied event gathered in a vector.
Events are split into 2 classes : the signals $S$ and the backgrounds $B$.
$m$ is the mixture coefficient between suignals and backgrounds.

$m$ can be seen as the probability for an event to be a signal $p(s)$. 
It naturally follows that $1-m$ is the probability for an event to be a background $p(b)=1-p(s)$.
\autoref{eq:mixture_model} can be written as
\begin{equation}
	p(x) = p(s)p(x|s) + p(b)p(x|b)
\end{equation}

In many interesting cases the nature of the event (signal/background) are not among the possible measurement that can be made.
Moreover the likelihoods $p(x|s)$ and $p(x|b)$ are intractable because of high dimension integrals or simply because computable formulas do not exist.
However building a simulator working only in forward mode allowing to sample from $p(x|m)$ is possible.
This problem is known as the inverse problem. 
The objective is to infer causal parameters from observations hence reversing the forward process which goes from causal parameters to observations.


The example that motivates this work is comming from High Energy Physics (HEP) where events are collisions in a particle accelerator.
Signals are events giving birth to a Higgs boson and backgrounds gather all the other collisions.
$m$ is therefore connected to branch factor (cross section ?) of the Higgs boson which is the parameter we want to measure.

\victor{Link between cross-section, branch factor, luminosity ?}

Measurements are made from a large bunch of independant and identically distributed events $D=\{x_i\}_{i=1}^N$.

\begin{align*}
	p(D|m) =& \prod_{i=1}^N m p(x|s) + (1-m) p(x|b) \\
	       =& \prod_{i=1}^N p(x|b) \left [(1-m) + m \frac{p(x|s)}{p(x|b)} \right ]\\
	       =& \underbrace{\left[ \prod_{i=1}^N p(x|b) \right ]}_{h(x)} \times 
	       \underbrace{\left [\prod_{i=1}^N (1-m) + m \frac{p(x|s)}{p(x|b)} \right ]}_{g_m(T(x))}
\end{align*}
with $T(x) = \frac{p(x|s)}{p(x|b)} $

Fisher-Neyman factorization theorem states that $T(x)$ is a sufficient summary statistic to obtain $m$

The maximum likelihood estimator, noted $\hat m$ is commonly used to estimate the parameter of interest.
Recall that maximum likelihood estimator are strictly equivalent to a maximum a posteriori estimator using a uniform prior.
This is a reasonable choice when we do not have prior knowledge as in this example.

\begin{equation}
	\hat m = \argmax_m p(m | D)
\end{equation}

It is more convenient to express the result as a deviation from the prediction of the Standard Model.
The deviation is defined as :

\begin{equation}
	\mu = \frac{p(s)}{p_{SM}(s)} = \frac{m}{p_{SM}(s)}
\end{equation}
$p_{SM}(s)$ is the expected probability to get a signal following the Standard Model.
Recovering $m$ from $\mu$ is trivially done with $m = \mu p_{SM}(s)$.

The estimator is now :
\begin{align}
	\hmu =& \argmax_\mu p(\mu | D) \\
	     =& \argmax_\mu \frac{p(\mu)}{p(D)} p(D | \mu) \\
	     =& \argmax_\mu p(\mu) p(D | \mu) \\
	     =& \argmax_\mu  p(D | \mu) \\
	     =& \argmax_\mu  \prod_{i=1}^N g_\mu(T(x)) \\
\end{align}


$T(x)$ can be obtained using a classifier $c$ trained to separate signals and backgrounds.
A Bayes optimal classifier output gives :
\begin{equation}
	c(x) = \frac{s p(x|s)}{(1-s) p(x|b) + s p(x|s)}
\end{equation}
where $s$ is the fraction of signals used in the training dataset.

\begin{equation}
	T(x) = \frac{c(x)}{(1-c(x))} \frac{(1-s)}{s} 
\end{equation}


Note : $c$ is also a sufficient summary statistic
\begin{equation}
	g_\mu(T(x)) = 1 - \mu p_{SM}(s) + \mu p_{SM}(s) \times \frac{c(x)}{(1-c(x))} \frac{(1-s)}{s} = f_\mu(c(x))
\end{equation}
