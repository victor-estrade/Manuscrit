%!TEX root = ../thesis.tex
%*******************************************************************************
%*********************************** First Chapter *****************************
%*******************************************************************************

\chapter{Data-driven science}  %Title of the First Chapter
\label{chap:intro_phy}
\ifpdf
    \graphicspath{{Chapter1/Figs/Raster/}{Chapter1/Figs/PDF/}{Chapter1/Figs/}}
\else
    \graphicspath{{Chapter1/Figs/Vector/}{Chapter1/Figs/}}
\fi

\cecile{This is a example of comment}

\victor{This is a example of comment}

\isabelle{This is a example of comment}


\section{Background}


\subsection{High energy physics} % (fold)
\label{sub:high_energy_physics}

\subsection{CERN} % (fold)
\label{sub:cern}


\subsection{Systematics} % (fold)
\label{sub:systematics}



\section{Motivation} % (fold)
\label{sec:motivation}

Limitations des méthodes actuelles

\subsection{Accuracy} % (fold)
\label{sub:accuracy}


\subsection{computation} % (fold)
\label{sub:computation}

\subsection{Understanding and methodology} % (fold)
\label{sub:understanding_and_methodology}

Levée de confusion entre mesure et découverte.

Évaluation quantitative empirique des résultats => test statistique "non parametrique"

+ justifications théorique
