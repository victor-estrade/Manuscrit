%!TEX root = ../thesis.tex
%*******************************************************************************
%****************************** Fourth Chapter *********************************
%*******************************************************************************

\chapter{Systematic aware learning}
\label{chap:systml}
\ifpdf
    \graphicspath{{Chapter4/Figs/Raster/}{Chapter4/Figs/PDF/}{Chapter4/Figs/}}
\else
    \graphicspath{{Chapter4/Figs/Vector/}{Chapter4/Figs/}}
\fi


Found this paper \url{https://arxiv.org/pdf/1903.10563.pdf} which review ML + Physics.

So let's dive into the state of the art of inference in Physics using ML.
Especially in the inverse problem setting.

\section{ TODO Papers }

\begin{itemize}
    \item \url{https://arxiv.org/abs/1506.02169} : Approximating Likelihood Ratios with Calibrated Discriminative Classifiers
    \item \url{https://arxiv.org/abs/1601.07913} : Parameterized Machine Learning for High-Energy Physics
    \item \url{https://arxiv.org/abs/1610.08328} : Event generator tuning using Bayesian optimization
    \item \url{https://arxiv.org/abs/1611.01046} : Learning to Pivot with Adversarial Networks
    \item \url{https://arxiv.org/abs/1706.04008} : Recurrent Inference Machines for Solving Inverse Problems
    \item \url{https://arxiv.org/abs/1707.07113} : Adversarial Variational Optimization of Non-Differentiable Simulators
    \item \url{https://arxiv.org/abs/1801.01497} : Massive optimal data compression and density estimation for scalable, likelihood-free inference in cosmology
    \item \url{https://arxiv.org/abs/1802.03537} : Automatic physical inference with information maximising neural networks
    https://arxiv.org/abs/1805.03961
    \item \url{https://arxiv.org/abs/1805.03961} : Study of constraint and impact of a nuisance parameter in maximum likelihood method
    \item \url{https://arxiv.org/abs/1805.07226} : Sequential Neural Likelihood: Fast Likelihood-free Inference with Autoregressive Flows
    \item \url{https://arxiv.org/abs/1805.00020} : A Guide to Constraining Effective Field Theories with Machine Learning
    \item \url{https://arxiv.org/abs/1805.00013} : Constraining Effective Field Theories with Machine Learning
    \item \url{https://arxiv.org/abs/1805.12244} : Mining gold from implicit models to improve likelihood-free inference
    \item \url{https://arxiv.org/abs/1806.11484} : Deep Learning and its Application to LHC Physics
    \item \url{https://arxiv.org/abs/1903.01473} : Nuisance hardened data compression for fast likelihood-free inference
    \item \url{} : 
    \item \url{} : 
    \item \url{} : 
    \item \url{https://cds.cern.ch/record/1099977/files/p111.pdf} : Computing Likelihood Functions for High-Energy Physics Experiments when Distributions are Defined by Simulators with Nuisance Parameters
    \item \url{https://arxiv.org/abs/1007.1727} : Asymptotic formulae for likelihood-based tests of new physics
    \item \url{https://www.pp.rhul.ac.uk/~cowan/stat/aachen/cowan_aachen14_1.pdf} : Cours de Glen 2012 part 1
    \item \url{https://www.pp.rhul.ac.uk/~cowan/stat/aachen/cowan_aachen14_2.pdf} : Cours de Glen 2012 part 2
    \item \url{https://www.pp.rhul.ac.uk/~cowan/stat/aachen/cowan_aachen14_3.pdf} : Cours de Glen 2012 part 3
    \item \url{https://www.pp.rhul.ac.uk/~cowan/stat/aachen/cowan_aachen14_4.pdf} : Cours de Glen 2012 part 4
    \item \url{https://www.pp.rhul.ac.uk/~cowan/stat/aachen/cowan_aachen14_5.pdf} : Cours de Glen 2012 part 5
    \item \url{https://arxiv.org/pdf/1503.07622.pdf} : Practical Statistics for the LHC
    \item \url{} : 
    \item \url{} : 
    \item \url{} : 
    \item \url{} : 
\end{itemize}


\section{ ABC }

Approximate Bayesian Computation is ...

\section{ Inferno }
\section{ Learning to become }

\section{ Mining gold }

Augment the simulated data with the likelihood ratio ...





